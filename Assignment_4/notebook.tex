
% Default to the notebook output style

    


% Inherit from the specified cell style.




    
\documentclass[11pt]{article}

    
    
    \usepackage[T1]{fontenc}
    % Nicer default font (+ math font) than Computer Modern for most use cases
    \usepackage{mathpazo}

    % Basic figure setup, for now with no caption control since it's done
    % automatically by Pandoc (which extracts ![](path) syntax from Markdown).
    \usepackage{graphicx}
    % We will generate all images so they have a width \maxwidth. This means
    % that they will get their normal width if they fit onto the page, but
    % are scaled down if they would overflow the margins.
    \makeatletter
    \def\maxwidth{\ifdim\Gin@nat@width>\linewidth\linewidth
    \else\Gin@nat@width\fi}
    \makeatother
    \let\Oldincludegraphics\includegraphics
    % Set max figure width to be 80% of text width, for now hardcoded.
    \renewcommand{\includegraphics}[1]{\Oldincludegraphics[width=.8\maxwidth]{#1}}
    % Ensure that by default, figures have no caption (until we provide a
    % proper Figure object with a Caption API and a way to capture that
    % in the conversion process - todo).
    \usepackage{caption}
    \DeclareCaptionLabelFormat{nolabel}{}
    \captionsetup{labelformat=nolabel}

    \usepackage{adjustbox} % Used to constrain images to a maximum size 
    \usepackage{xcolor} % Allow colors to be defined
    \usepackage{enumerate} % Needed for markdown enumerations to work
    \usepackage{geometry} % Used to adjust the document margins
    \usepackage{amsmath} % Equations
    \usepackage{amssymb} % Equations
    \usepackage{textcomp} % defines textquotesingle
    % Hack from http://tex.stackexchange.com/a/47451/13684:
    \AtBeginDocument{%
        \def\PYZsq{\textquotesingle}% Upright quotes in Pygmentized code
    }
    \usepackage{upquote} % Upright quotes for verbatim code
    \usepackage{eurosym} % defines \euro
    \usepackage[mathletters]{ucs} % Extended unicode (utf-8) support
    \usepackage[utf8x]{inputenc} % Allow utf-8 characters in the tex document
    \usepackage{fancyvrb} % verbatim replacement that allows latex
    \usepackage{grffile} % extends the file name processing of package graphics 
                         % to support a larger range 
    % The hyperref package gives us a pdf with properly built
    % internal navigation ('pdf bookmarks' for the table of contents,
    % internal cross-reference links, web links for URLs, etc.)
    \usepackage{hyperref}
    \usepackage{longtable} % longtable support required by pandoc >1.10
    \usepackage{booktabs}  % table support for pandoc > 1.12.2
    \usepackage[inline]{enumitem} % IRkernel/repr support (it uses the enumerate* environment)
    \usepackage[normalem]{ulem} % ulem is needed to support strikethroughs (\sout)
                                % normalem makes italics be italics, not underlines
    

    
    
    % Colors for the hyperref package
    \definecolor{urlcolor}{rgb}{0,.145,.698}
    \definecolor{linkcolor}{rgb}{.71,0.21,0.01}
    \definecolor{citecolor}{rgb}{.12,.54,.11}

    % ANSI colors
    \definecolor{ansi-black}{HTML}{3E424D}
    \definecolor{ansi-black-intense}{HTML}{282C36}
    \definecolor{ansi-red}{HTML}{E75C58}
    \definecolor{ansi-red-intense}{HTML}{B22B31}
    \definecolor{ansi-green}{HTML}{00A250}
    \definecolor{ansi-green-intense}{HTML}{007427}
    \definecolor{ansi-yellow}{HTML}{DDB62B}
    \definecolor{ansi-yellow-intense}{HTML}{B27D12}
    \definecolor{ansi-blue}{HTML}{208FFB}
    \definecolor{ansi-blue-intense}{HTML}{0065CA}
    \definecolor{ansi-magenta}{HTML}{D160C4}
    \definecolor{ansi-magenta-intense}{HTML}{A03196}
    \definecolor{ansi-cyan}{HTML}{60C6C8}
    \definecolor{ansi-cyan-intense}{HTML}{258F8F}
    \definecolor{ansi-white}{HTML}{C5C1B4}
    \definecolor{ansi-white-intense}{HTML}{A1A6B2}

    % commands and environments needed by pandoc snippets
    % extracted from the output of `pandoc -s`
    \providecommand{\tightlist}{%
      \setlength{\itemsep}{0pt}\setlength{\parskip}{0pt}}
    \DefineVerbatimEnvironment{Highlighting}{Verbatim}{commandchars=\\\{\}}
    % Add ',fontsize=\small' for more characters per line
    \newenvironment{Shaded}{}{}
    \newcommand{\KeywordTok}[1]{\textcolor[rgb]{0.00,0.44,0.13}{\textbf{{#1}}}}
    \newcommand{\DataTypeTok}[1]{\textcolor[rgb]{0.56,0.13,0.00}{{#1}}}
    \newcommand{\DecValTok}[1]{\textcolor[rgb]{0.25,0.63,0.44}{{#1}}}
    \newcommand{\BaseNTok}[1]{\textcolor[rgb]{0.25,0.63,0.44}{{#1}}}
    \newcommand{\FloatTok}[1]{\textcolor[rgb]{0.25,0.63,0.44}{{#1}}}
    \newcommand{\CharTok}[1]{\textcolor[rgb]{0.25,0.44,0.63}{{#1}}}
    \newcommand{\StringTok}[1]{\textcolor[rgb]{0.25,0.44,0.63}{{#1}}}
    \newcommand{\CommentTok}[1]{\textcolor[rgb]{0.38,0.63,0.69}{\textit{{#1}}}}
    \newcommand{\OtherTok}[1]{\textcolor[rgb]{0.00,0.44,0.13}{{#1}}}
    \newcommand{\AlertTok}[1]{\textcolor[rgb]{1.00,0.00,0.00}{\textbf{{#1}}}}
    \newcommand{\FunctionTok}[1]{\textcolor[rgb]{0.02,0.16,0.49}{{#1}}}
    \newcommand{\RegionMarkerTok}[1]{{#1}}
    \newcommand{\ErrorTok}[1]{\textcolor[rgb]{1.00,0.00,0.00}{\textbf{{#1}}}}
    \newcommand{\NormalTok}[1]{{#1}}
    
    % Additional commands for more recent versions of Pandoc
    \newcommand{\ConstantTok}[1]{\textcolor[rgb]{0.53,0.00,0.00}{{#1}}}
    \newcommand{\SpecialCharTok}[1]{\textcolor[rgb]{0.25,0.44,0.63}{{#1}}}
    \newcommand{\VerbatimStringTok}[1]{\textcolor[rgb]{0.25,0.44,0.63}{{#1}}}
    \newcommand{\SpecialStringTok}[1]{\textcolor[rgb]{0.73,0.40,0.53}{{#1}}}
    \newcommand{\ImportTok}[1]{{#1}}
    \newcommand{\DocumentationTok}[1]{\textcolor[rgb]{0.73,0.13,0.13}{\textit{{#1}}}}
    \newcommand{\AnnotationTok}[1]{\textcolor[rgb]{0.38,0.63,0.69}{\textbf{\textit{{#1}}}}}
    \newcommand{\CommentVarTok}[1]{\textcolor[rgb]{0.38,0.63,0.69}{\textbf{\textit{{#1}}}}}
    \newcommand{\VariableTok}[1]{\textcolor[rgb]{0.10,0.09,0.49}{{#1}}}
    \newcommand{\ControlFlowTok}[1]{\textcolor[rgb]{0.00,0.44,0.13}{\textbf{{#1}}}}
    \newcommand{\OperatorTok}[1]{\textcolor[rgb]{0.40,0.40,0.40}{{#1}}}
    \newcommand{\BuiltInTok}[1]{{#1}}
    \newcommand{\ExtensionTok}[1]{{#1}}
    \newcommand{\PreprocessorTok}[1]{\textcolor[rgb]{0.74,0.48,0.00}{{#1}}}
    \newcommand{\AttributeTok}[1]{\textcolor[rgb]{0.49,0.56,0.16}{{#1}}}
    \newcommand{\InformationTok}[1]{\textcolor[rgb]{0.38,0.63,0.69}{\textbf{\textit{{#1}}}}}
    \newcommand{\WarningTok}[1]{\textcolor[rgb]{0.38,0.63,0.69}{\textbf{\textit{{#1}}}}}
    
    
    % Define a nice break command that doesn't care if a line doesn't already
    % exist.
    \def\br{\hspace*{\fill} \\* }
    % Math Jax compatability definitions
    \def\gt{>}
    \def\lt{<}
    % Document parameters
    \title{Assignment4}
    
    
    

    % Pygments definitions
    
\makeatletter
\def\PY@reset{\let\PY@it=\relax \let\PY@bf=\relax%
    \let\PY@ul=\relax \let\PY@tc=\relax%
    \let\PY@bc=\relax \let\PY@ff=\relax}
\def\PY@tok#1{\csname PY@tok@#1\endcsname}
\def\PY@toks#1+{\ifx\relax#1\empty\else%
    \PY@tok{#1}\expandafter\PY@toks\fi}
\def\PY@do#1{\PY@bc{\PY@tc{\PY@ul{%
    \PY@it{\PY@bf{\PY@ff{#1}}}}}}}
\def\PY#1#2{\PY@reset\PY@toks#1+\relax+\PY@do{#2}}

\expandafter\def\csname PY@tok@gt\endcsname{\def\PY@tc##1{\textcolor[rgb]{0.00,0.27,0.87}{##1}}}
\expandafter\def\csname PY@tok@ow\endcsname{\let\PY@bf=\textbf\def\PY@tc##1{\textcolor[rgb]{0.67,0.13,1.00}{##1}}}
\expandafter\def\csname PY@tok@s1\endcsname{\def\PY@tc##1{\textcolor[rgb]{0.73,0.13,0.13}{##1}}}
\expandafter\def\csname PY@tok@kn\endcsname{\let\PY@bf=\textbf\def\PY@tc##1{\textcolor[rgb]{0.00,0.50,0.00}{##1}}}
\expandafter\def\csname PY@tok@ge\endcsname{\let\PY@it=\textit}
\expandafter\def\csname PY@tok@si\endcsname{\let\PY@bf=\textbf\def\PY@tc##1{\textcolor[rgb]{0.73,0.40,0.53}{##1}}}
\expandafter\def\csname PY@tok@nf\endcsname{\def\PY@tc##1{\textcolor[rgb]{0.00,0.00,1.00}{##1}}}
\expandafter\def\csname PY@tok@mi\endcsname{\def\PY@tc##1{\textcolor[rgb]{0.40,0.40,0.40}{##1}}}
\expandafter\def\csname PY@tok@err\endcsname{\def\PY@bc##1{\setlength{\fboxsep}{0pt}\fcolorbox[rgb]{1.00,0.00,0.00}{1,1,1}{\strut ##1}}}
\expandafter\def\csname PY@tok@bp\endcsname{\def\PY@tc##1{\textcolor[rgb]{0.00,0.50,0.00}{##1}}}
\expandafter\def\csname PY@tok@gd\endcsname{\def\PY@tc##1{\textcolor[rgb]{0.63,0.00,0.00}{##1}}}
\expandafter\def\csname PY@tok@mf\endcsname{\def\PY@tc##1{\textcolor[rgb]{0.40,0.40,0.40}{##1}}}
\expandafter\def\csname PY@tok@ni\endcsname{\let\PY@bf=\textbf\def\PY@tc##1{\textcolor[rgb]{0.60,0.60,0.60}{##1}}}
\expandafter\def\csname PY@tok@kp\endcsname{\def\PY@tc##1{\textcolor[rgb]{0.00,0.50,0.00}{##1}}}
\expandafter\def\csname PY@tok@c1\endcsname{\let\PY@it=\textit\def\PY@tc##1{\textcolor[rgb]{0.25,0.50,0.50}{##1}}}
\expandafter\def\csname PY@tok@vc\endcsname{\def\PY@tc##1{\textcolor[rgb]{0.10,0.09,0.49}{##1}}}
\expandafter\def\csname PY@tok@kd\endcsname{\let\PY@bf=\textbf\def\PY@tc##1{\textcolor[rgb]{0.00,0.50,0.00}{##1}}}
\expandafter\def\csname PY@tok@nl\endcsname{\def\PY@tc##1{\textcolor[rgb]{0.63,0.63,0.00}{##1}}}
\expandafter\def\csname PY@tok@nt\endcsname{\let\PY@bf=\textbf\def\PY@tc##1{\textcolor[rgb]{0.00,0.50,0.00}{##1}}}
\expandafter\def\csname PY@tok@gi\endcsname{\def\PY@tc##1{\textcolor[rgb]{0.00,0.63,0.00}{##1}}}
\expandafter\def\csname PY@tok@nn\endcsname{\let\PY@bf=\textbf\def\PY@tc##1{\textcolor[rgb]{0.00,0.00,1.00}{##1}}}
\expandafter\def\csname PY@tok@mo\endcsname{\def\PY@tc##1{\textcolor[rgb]{0.40,0.40,0.40}{##1}}}
\expandafter\def\csname PY@tok@ne\endcsname{\let\PY@bf=\textbf\def\PY@tc##1{\textcolor[rgb]{0.82,0.25,0.23}{##1}}}
\expandafter\def\csname PY@tok@sc\endcsname{\def\PY@tc##1{\textcolor[rgb]{0.73,0.13,0.13}{##1}}}
\expandafter\def\csname PY@tok@s\endcsname{\def\PY@tc##1{\textcolor[rgb]{0.73,0.13,0.13}{##1}}}
\expandafter\def\csname PY@tok@no\endcsname{\def\PY@tc##1{\textcolor[rgb]{0.53,0.00,0.00}{##1}}}
\expandafter\def\csname PY@tok@sb\endcsname{\def\PY@tc##1{\textcolor[rgb]{0.73,0.13,0.13}{##1}}}
\expandafter\def\csname PY@tok@il\endcsname{\def\PY@tc##1{\textcolor[rgb]{0.40,0.40,0.40}{##1}}}
\expandafter\def\csname PY@tok@ss\endcsname{\def\PY@tc##1{\textcolor[rgb]{0.10,0.09,0.49}{##1}}}
\expandafter\def\csname PY@tok@sd\endcsname{\let\PY@it=\textit\def\PY@tc##1{\textcolor[rgb]{0.73,0.13,0.13}{##1}}}
\expandafter\def\csname PY@tok@na\endcsname{\def\PY@tc##1{\textcolor[rgb]{0.49,0.56,0.16}{##1}}}
\expandafter\def\csname PY@tok@kr\endcsname{\let\PY@bf=\textbf\def\PY@tc##1{\textcolor[rgb]{0.00,0.50,0.00}{##1}}}
\expandafter\def\csname PY@tok@gs\endcsname{\let\PY@bf=\textbf}
\expandafter\def\csname PY@tok@cp\endcsname{\def\PY@tc##1{\textcolor[rgb]{0.74,0.48,0.00}{##1}}}
\expandafter\def\csname PY@tok@kc\endcsname{\let\PY@bf=\textbf\def\PY@tc##1{\textcolor[rgb]{0.00,0.50,0.00}{##1}}}
\expandafter\def\csname PY@tok@c\endcsname{\let\PY@it=\textit\def\PY@tc##1{\textcolor[rgb]{0.25,0.50,0.50}{##1}}}
\expandafter\def\csname PY@tok@m\endcsname{\def\PY@tc##1{\textcolor[rgb]{0.40,0.40,0.40}{##1}}}
\expandafter\def\csname PY@tok@vi\endcsname{\def\PY@tc##1{\textcolor[rgb]{0.10,0.09,0.49}{##1}}}
\expandafter\def\csname PY@tok@vg\endcsname{\def\PY@tc##1{\textcolor[rgb]{0.10,0.09,0.49}{##1}}}
\expandafter\def\csname PY@tok@sx\endcsname{\def\PY@tc##1{\textcolor[rgb]{0.00,0.50,0.00}{##1}}}
\expandafter\def\csname PY@tok@sh\endcsname{\def\PY@tc##1{\textcolor[rgb]{0.73,0.13,0.13}{##1}}}
\expandafter\def\csname PY@tok@nb\endcsname{\def\PY@tc##1{\textcolor[rgb]{0.00,0.50,0.00}{##1}}}
\expandafter\def\csname PY@tok@k\endcsname{\let\PY@bf=\textbf\def\PY@tc##1{\textcolor[rgb]{0.00,0.50,0.00}{##1}}}
\expandafter\def\csname PY@tok@o\endcsname{\def\PY@tc##1{\textcolor[rgb]{0.40,0.40,0.40}{##1}}}
\expandafter\def\csname PY@tok@nd\endcsname{\def\PY@tc##1{\textcolor[rgb]{0.67,0.13,1.00}{##1}}}
\expandafter\def\csname PY@tok@w\endcsname{\def\PY@tc##1{\textcolor[rgb]{0.73,0.73,0.73}{##1}}}
\expandafter\def\csname PY@tok@s2\endcsname{\def\PY@tc##1{\textcolor[rgb]{0.73,0.13,0.13}{##1}}}
\expandafter\def\csname PY@tok@ch\endcsname{\let\PY@it=\textit\def\PY@tc##1{\textcolor[rgb]{0.25,0.50,0.50}{##1}}}
\expandafter\def\csname PY@tok@mh\endcsname{\def\PY@tc##1{\textcolor[rgb]{0.40,0.40,0.40}{##1}}}
\expandafter\def\csname PY@tok@go\endcsname{\def\PY@tc##1{\textcolor[rgb]{0.53,0.53,0.53}{##1}}}
\expandafter\def\csname PY@tok@kt\endcsname{\def\PY@tc##1{\textcolor[rgb]{0.69,0.00,0.25}{##1}}}
\expandafter\def\csname PY@tok@gp\endcsname{\let\PY@bf=\textbf\def\PY@tc##1{\textcolor[rgb]{0.00,0.00,0.50}{##1}}}
\expandafter\def\csname PY@tok@cpf\endcsname{\let\PY@it=\textit\def\PY@tc##1{\textcolor[rgb]{0.25,0.50,0.50}{##1}}}
\expandafter\def\csname PY@tok@cm\endcsname{\let\PY@it=\textit\def\PY@tc##1{\textcolor[rgb]{0.25,0.50,0.50}{##1}}}
\expandafter\def\csname PY@tok@gu\endcsname{\let\PY@bf=\textbf\def\PY@tc##1{\textcolor[rgb]{0.50,0.00,0.50}{##1}}}
\expandafter\def\csname PY@tok@mb\endcsname{\def\PY@tc##1{\textcolor[rgb]{0.40,0.40,0.40}{##1}}}
\expandafter\def\csname PY@tok@nc\endcsname{\let\PY@bf=\textbf\def\PY@tc##1{\textcolor[rgb]{0.00,0.00,1.00}{##1}}}
\expandafter\def\csname PY@tok@gh\endcsname{\let\PY@bf=\textbf\def\PY@tc##1{\textcolor[rgb]{0.00,0.00,0.50}{##1}}}
\expandafter\def\csname PY@tok@gr\endcsname{\def\PY@tc##1{\textcolor[rgb]{1.00,0.00,0.00}{##1}}}
\expandafter\def\csname PY@tok@cs\endcsname{\let\PY@it=\textit\def\PY@tc##1{\textcolor[rgb]{0.25,0.50,0.50}{##1}}}
\expandafter\def\csname PY@tok@se\endcsname{\let\PY@bf=\textbf\def\PY@tc##1{\textcolor[rgb]{0.73,0.40,0.13}{##1}}}
\expandafter\def\csname PY@tok@nv\endcsname{\def\PY@tc##1{\textcolor[rgb]{0.10,0.09,0.49}{##1}}}
\expandafter\def\csname PY@tok@sr\endcsname{\def\PY@tc##1{\textcolor[rgb]{0.73,0.40,0.53}{##1}}}

\def\PYZbs{\char`\\}
\def\PYZus{\char`\_}
\def\PYZob{\char`\{}
\def\PYZcb{\char`\}}
\def\PYZca{\char`\^}
\def\PYZam{\char`\&}
\def\PYZlt{\char`\<}
\def\PYZgt{\char`\>}
\def\PYZsh{\char`\#}
\def\PYZpc{\char`\%}
\def\PYZdl{\char`\$}
\def\PYZhy{\char`\-}
\def\PYZsq{\char`\'}
\def\PYZdq{\char`\"}
\def\PYZti{\char`\~}
% for compatibility with earlier versions
\def\PYZat{@}
\def\PYZlb{[}
\def\PYZrb{]}
\makeatother


    % Exact colors from NB
    \definecolor{incolor}{rgb}{0.0, 0.0, 0.5}
    \definecolor{outcolor}{rgb}{0.545, 0.0, 0.0}



    
    % Prevent overflowing lines due to hard-to-break entities
    \sloppy 
    % Setup hyperref package
    \hypersetup{
      breaklinks=true,  % so long urls are correctly broken across lines
      colorlinks=true,
      urlcolor=urlcolor,
      linkcolor=linkcolor,
      citecolor=citecolor,
      }
    % Slightly bigger margins than the latex defaults
    
    \geometry{verbose,tmargin=1in,bmargin=1in,lmargin=1in,rmargin=1in}
    
    

    \begin{document}
    
    
    \maketitle
    
    

    
    \section{\texorpdfstring{ \emph{Assignment
4}}{ Assignment 4}}\label{assignment-4}

\begin{center}\rule{0.5\linewidth}{\linethickness}\end{center}

\subsubsection{\texorpdfstring{\emph{Main
Task}}{Main Task}}\label{main-task}

\begin{quote}
To compute and plot the Low Level Wind Shear (LLWS) by computing
windspeed from a wrfout NetCDF file (Alaska or Florida), using V and F
fields .
\end{quote}

\subsubsection{\texorpdfstring{\emph{Sub
Tasks}}{Sub Tasks}}\label{sub-tasks}

 \textgreater{} 1. 3D U and V fields are on ``staggered'' grids. Use the
function destagger\_uv(ustagger, vstagger) \textgreater{} 2. Covert XLAT
\& XLON to map coordinates \textgreater{} 3. Plot LLWS(knots) using
Basemap(with Alaska or Florida Map) \textgreater{} 4. Plot a warning
area where LLWS exceeds 10 knots * 

    \subsubsection{\texorpdfstring{ \emph{Define}
\texttt{WRFOUT\_FILE\_PATH}}{ Define  WRFOUT\_FILE\_PATH}}\label{define-wrfoutux5ffileux5fpath}

    \begin{Verbatim}[commandchars=\\\{\}]
{\color{incolor}In [{\color{incolor}1}]:} \PY{k+kn}{import} \PY{n+nn}{numpy} \PY{k}{as} \PY{n+nn}{np}
        \PY{k+kn}{import} \PY{n+nn}{matplotlib}\PY{n+nn}{.}\PY{n+nn}{pyplot} \PY{k}{as} \PY{n+nn}{plt}
        \PY{k+kn}{import} \PY{n+nn}{mpl\PYZus{}toolkits}\PY{n+nn}{.}\PY{n+nn}{basemap} \PY{k}{as} \PY{n+nn}{bm}
        \PY{k+kn}{import} \PY{n+nn}{netCDF4}
        
        \PY{c+c1}{\PYZsh{}WRFOUT\PYZus{}FILE\PYZus{}PATH = \PYZsq{}wrfout\PYZus{}d01\PYZus{}2017\PYZhy{}09\PYZhy{}11\PYZus{}06:00:00\PYZsq{} \PYZsh{} Florida}
        \PY{n}{WRFOUT\PYZus{}FILE\PYZus{}PATH} \PY{o}{=}\PY{l+s+s1}{\PYZsq{}}\PY{l+s+s1}{wrfout\PYZus{}d01\PYZus{}2018\PYZhy{}08\PYZhy{}13\PYZus{}00:00:00}\PY{l+s+s1}{\PYZsq{}} \PY{c+c1}{\PYZsh{}Alaska}
\end{Verbatim}


    \subsubsection{\texorpdfstring{ \emph{Extract}
\texttt{U\ V\ XLAT\ XLONG} \emph{destaggerd}
\texttt{U\ \&\ V}}{ Extract U V XLAT XLONG destaggerd U \& V}}\label{extract-u-v-xlat-xlong-destaggerd-u-v}

    \begin{Verbatim}[commandchars=\\\{\}]
{\color{incolor}In [{\color{incolor}2}]:} \PY{n}{dataset} \PY{o}{=} \PY{n}{netCDF4}\PY{o}{.}\PY{n}{Dataset}\PY{p}{(}\PY{n}{WRFOUT\PYZus{}FILE\PYZus{}PATH}\PY{p}{,} \PY{l+s+s1}{\PYZsq{}}\PY{l+s+s1}{r}\PY{l+s+s1}{\PYZsq{}}\PY{p}{)}
        
        \PY{c+c1}{\PYZsh{} Get the 2D arrays of lats and lons, corresponding to }
        \PY{c+c1}{\PYZsh{} each data grid point}
        \PY{c+c1}{\PYZsh{} Recall that dimensions ar (Time, south\PYZus{}north, west\PYZus{}east)}
        
        \PY{n}{XLAT} \PY{o}{=} \PY{n}{dataset}\PY{o}{.}\PY{n}{variables}\PY{p}{[}\PY{l+s+s1}{\PYZsq{}}\PY{l+s+s1}{XLAT}\PY{l+s+s1}{\PYZsq{}}\PY{p}{]}\PY{p}{[}\PY{p}{:}\PY{p}{]}
        \PY{n}{XLON} \PY{o}{=} \PY{n}{dataset}\PY{o}{.}\PY{n}{variables}\PY{p}{[}\PY{l+s+s1}{\PYZsq{}}\PY{l+s+s1}{XLONG}\PY{l+s+s1}{\PYZsq{}}\PY{p}{]}\PY{p}{[}\PY{p}{:}\PY{p}{]}
        
        \PY{c+c1}{\PYZsh{}Get the lower left and upper right corner lat and lon}
        \PY{n}{ll\PYZus{}lat} \PY{o}{=} \PY{n}{XLAT}\PY{p}{[}\PY{l+m+mi}{0}\PY{p}{,} \PY{l+m+mi}{0}\PY{p}{,} \PY{l+m+mi}{0}\PY{p}{]}\PY{p}{;} \PY{n}{ll\PYZus{}lon} \PY{o}{=} \PY{n}{XLON}\PY{p}{[}\PY{l+m+mi}{0}\PY{p}{,} \PY{l+m+mi}{0}\PY{p}{,} \PY{l+m+mi}{0}\PY{p}{]}
        \PY{n}{ur\PYZus{}lat} \PY{o}{=} \PY{n}{XLAT}\PY{p}{[}\PY{l+m+mi}{0}\PY{p}{,} \PY{o}{\PYZhy{}}\PY{l+m+mi}{1}\PY{p}{,} \PY{o}{\PYZhy{}}\PY{l+m+mi}{1}\PY{p}{]}\PY{p}{;} \PY{n}{ur\PYZus{}lon} \PY{o}{=} \PY{n}{XLON}\PY{p}{[}\PY{l+m+mi}{0}\PY{p}{,} \PY{o}{\PYZhy{}}\PY{l+m+mi}{1}\PY{p}{,} \PY{o}{\PYZhy{}}\PY{l+m+mi}{1}\PY{p}{]}
        
        
        
        \PY{k}{def} \PY{n+nf}{destagger\PYZus{}uv}\PY{p}{(}\PY{n}{ustagger}\PY{o}{=}\PY{k+kc}{None}\PY{p}{,} \PY{n}{vstagger}\PY{o}{=}\PY{k+kc}{None}\PY{p}{)}\PY{p}{:}
           
            \PY{l+s+sd}{\PYZdq{}\PYZdq{}\PYZdq{}}
        \PY{l+s+sd}{    Creates arrays u and v on grid points rather than their native mass points}
        \PY{l+s+sd}{    by assigning to each grid point the average of the adjacent mass points}
        
        \PY{l+s+sd}{    Assumes ustagger has dimensions (Time, bottom\PYZus{}top, south\PYZus{}north, west\PYZus{}east\PYZus{}stag)}
        \PY{l+s+sd}{    Assumes vstagger has dimensions (Time, bottom\PYZus{}top, south\PYZus{}north\PYZus{}stag, west\PYZus{}east)}
        
        \PY{l+s+sd}{    By averaging, creates U and V arrays of dimensions }
        \PY{l+s+sd}{    (Time, bottom\PYZus{}top, south\PYZus{}north, west\PYZus{}east)}
        \PY{l+s+sd}{    \PYZdq{}\PYZdq{}\PYZdq{}}
        
            \PY{c+c1}{\PYZsh{} We can compute the dimensions of the arrays through knowledge of }
            \PY{c+c1}{\PYZsh{} their shapes}
            \PY{n}{Time\PYZus{}dim} \PY{o}{=} \PY{n}{ustagger}\PY{o}{.}\PY{n}{shape}\PY{p}{[}\PY{l+m+mi}{0}\PY{p}{]}
            \PY{n}{bottom\PYZus{}top\PYZus{}dim} \PY{o}{=} \PY{n}{ustagger}\PY{o}{.}\PY{n}{shape}\PY{p}{[}\PY{l+m+mi}{1}\PY{p}{]}
            \PY{n}{south\PYZus{}north\PYZus{}dim} \PY{o}{=} \PY{n}{ustagger}\PY{o}{.}\PY{n}{shape}\PY{p}{[}\PY{l+m+mi}{2}\PY{p}{]}
            \PY{n}{west\PYZus{}east\PYZus{}stag\PYZus{}dim} \PY{o}{=} \PY{n}{ustagger}\PY{o}{.}\PY{n}{shape}\PY{p}{[}\PY{l+m+mi}{3}\PY{p}{]}
            \PY{n}{south\PYZus{}north\PYZus{}stag\PYZus{}dim} \PY{o}{=} \PY{n}{vstagger}\PY{o}{.}\PY{n}{shape}\PY{p}{[}\PY{l+m+mi}{2}\PY{p}{]}
            \PY{n}{west\PYZus{}east\PYZus{}dim} \PY{o}{=} \PY{n}{vstagger}\PY{o}{.}\PY{n}{shape}\PY{p}{[}\PY{l+m+mi}{3}\PY{p}{]}
        
            \PY{c+c1}{\PYZsh{} Allocate and shape the arrays that will be returned}
            \PY{n}{u} \PY{o}{=} \PY{n}{v} \PY{o}{=} \PY{n}{np}\PY{o}{.}\PY{n}{ndarray}\PY{p}{(} \PY{p}{(}\PY{n}{Time\PYZus{}dim}\PY{p}{,} \PY{n}{bottom\PYZus{}top\PYZus{}dim}\PY{p}{,} \PY{n}{south\PYZus{}north\PYZus{}dim}\PY{p}{,} \PY{n}{west\PYZus{}east\PYZus{}dim}\PY{p}{)} \PY{p}{)}
        
            \PY{c+c1}{\PYZsh{} Now the destaggering \PYZhy{} each grid point in the destaggered array is}
            \PY{c+c1}{\PYZsh{} the average of the adjacent mass points in the staggered arrays}
        
            \PY{k}{for} \PY{n}{j} \PY{o+ow}{in} \PY{n}{np}\PY{o}{.}\PY{n}{arange}\PY{p}{(}\PY{n}{west\PYZus{}east\PYZus{}dim}\PY{p}{)}\PY{p}{:}
                \PY{n}{u}\PY{p}{[}\PY{p}{:}\PY{p}{,}\PY{p}{:}\PY{p}{,}\PY{p}{:}\PY{p}{,}\PY{n}{j}\PY{p}{]} \PY{o}{=} \PY{p}{(} \PY{n}{ustagger}\PY{p}{[}\PY{p}{:}\PY{p}{,}\PY{p}{:}\PY{p}{,}\PY{p}{:}\PY{p}{,}\PY{n}{j}\PY{p}{]} \PY{o}{+} \PY{n}{ustagger}\PY{p}{[}\PY{p}{:}\PY{p}{,}\PY{p}{:}\PY{p}{,}\PY{p}{:}\PY{p}{,}\PY{n}{j}\PY{o}{+}\PY{l+m+mi}{1}\PY{p}{]} \PY{p}{)} \PY{o}{/} \PY{l+m+mf}{2.0}
            \PY{k}{for} \PY{n}{i} \PY{o+ow}{in} \PY{n}{np}\PY{o}{.}\PY{n}{arange}\PY{p}{(}\PY{n}{south\PYZus{}north\PYZus{}dim}\PY{p}{)}\PY{p}{:}
                \PY{n}{v}\PY{p}{[}\PY{p}{:}\PY{p}{,}\PY{p}{:}\PY{p}{,}\PY{n}{i}\PY{p}{,}\PY{p}{:}\PY{p}{]} \PY{o}{=} \PY{p}{(} \PY{n}{vstagger}\PY{p}{[}\PY{p}{:}\PY{p}{,}\PY{p}{:}\PY{p}{,}\PY{n}{i}\PY{p}{,}\PY{p}{:}\PY{p}{]} \PY{o}{+} \PY{n}{vstagger}\PY{p}{[}\PY{p}{:}\PY{p}{,}\PY{p}{:}\PY{p}{,}\PY{n}{i}\PY{o}{+}\PY{l+m+mi}{1}\PY{p}{,}\PY{p}{:}\PY{p}{]} \PY{p}{)} \PY{o}{/} \PY{l+m+mf}{2.0}    
        
            \PY{k}{return} \PY{n}{u}\PY{p}{,} \PY{n}{v}
        
        \PY{c+c1}{\PYZsh{} This is all we need to define a cylindrical map project}
        \PY{n}{the\PYZus{}map} \PY{o}{=} \PY{n}{bm}\PY{o}{.}\PY{n}{Basemap}\PY{p}{(}\PY{n}{projection}\PY{o}{=}\PY{l+s+s1}{\PYZsq{}}\PY{l+s+s1}{cyl}\PY{l+s+s1}{\PYZsq{}}\PY{p}{,}
                            \PY{n}{llcrnrlon}\PY{o}{=}\PY{n}{ll\PYZus{}lon}\PY{p}{,}
                            \PY{n}{llcrnrlat}\PY{o}{=}\PY{n}{ll\PYZus{}lat}\PY{p}{,}
                            \PY{n}{urcrnrlon}\PY{o}{=}\PY{n}{ur\PYZus{}lon}\PY{p}{,}
                            \PY{n}{urcrnrlat}\PY{o}{=}\PY{n}{ur\PYZus{}lat}\PY{p}{,}
                            \PY{n}{resolution}\PY{o}{=} \PY{l+s+s1}{\PYZsq{}}\PY{l+s+s1}{i}\PY{l+s+s1}{\PYZsq{}}\PY{p}{,}
                            \PY{n}{area\PYZus{}thresh}\PY{o}{=}\PY{l+m+mf}{100.0}\PY{p}{)}
        \PY{n}{the\PYZus{}map}\PY{o}{.}\PY{n}{drawcoastlines}\PY{p}{(}\PY{p}{)}
        \PY{n}{U\PYZus{}staggered} \PY{o}{=} \PY{n}{dataset}\PY{o}{.}\PY{n}{variables}\PY{p}{[}\PY{l+s+s1}{\PYZsq{}}\PY{l+s+s1}{U}\PY{l+s+s1}{\PYZsq{}}\PY{p}{]}\PY{p}{[}\PY{p}{:}\PY{p}{]}
        \PY{n}{V\PYZus{}staggered} \PY{o}{=} \PY{n}{dataset}\PY{o}{.}\PY{n}{variables}\PY{p}{[}\PY{l+s+s1}{\PYZsq{}}\PY{l+s+s1}{V}\PY{l+s+s1}{\PYZsq{}}\PY{p}{]}\PY{p}{[}\PY{p}{:}\PY{p}{]}
        \PY{n}{U}\PY{p}{,} \PY{n}{V} \PY{o}{=} \PY{n}{destagger\PYZus{}uv}\PY{p}{(}\PY{n}{U\PYZus{}staggered}\PY{p}{,} \PY{n}{V\PYZus{}staggered}\PY{p}{)}
        
        \PY{c+c1}{\PYZsh{}Create WS10}
        \PY{n}{WS} \PY{o}{=} \PY{p}{(}\PY{n}{np}\PY{o}{.}\PY{n}{sqrt}\PY{p}{(}\PY{n}{U}\PY{o}{*}\PY{o}{*}\PY{l+m+mi}{2} \PY{o}{+} \PY{n}{V}\PY{o}{*}\PY{o}{*}\PY{l+m+mi}{2}\PY{p}{)}\PY{p}{)} \PY{o}{*}\PY{l+m+mf}{1.94384} \PY{c+c1}{\PYZsh{}convert the units to knots}
        \PY{n}{dataset}\PY{o}{.}\PY{n}{close}\PY{p}{(}\PY{p}{)}
\end{Verbatim}


    \begin{center}
    \adjustimage{max size={0.9\linewidth}{0.9\paperheight}}{output_4_0.png}
    \end{center}
    { \hspace*{\fill} \\}
    
    \subsubsection{\texorpdfstring{ \emph{Compute \texttt{LLWS} 2D field and
plot it}
}{ Compute LLWS 2D field and plot it }}\label{compute-llws-2d-field-and-plot-it}

    \begin{Verbatim}[commandchars=\\\{\}]
{\color{incolor}In [{\color{incolor}3}]:} \PY{n}{LLWS} \PY{o}{=} \PY{n+nb}{abs}\PY{p}{(}\PY{n}{WS}\PY{p}{[}\PY{l+m+mi}{0}\PY{p}{,}\PY{l+m+mi}{1}\PY{p}{,}\PY{p}{:}\PY{p}{,}\PY{p}{:}\PY{p}{]} \PY{o}{\PYZhy{}} \PY{n}{WS}\PY{p}{[}\PY{l+m+mi}{0}\PY{p}{,}\PY{l+m+mi}{0}\PY{p}{,}\PY{p}{:}\PY{p}{,}\PY{p}{:}\PY{p}{]}\PY{p}{)}
        
        \PY{c+c1}{\PYZsh{}2d map coordinates}
        \PY{n}{map2d\PYZus{}x}\PY{p}{,} \PY{n}{map2d\PYZus{}y} \PY{o}{=} \PY{n}{the\PYZus{}map}\PY{p}{(}\PY{n}{XLON}\PY{p}{[}\PY{l+m+mi}{0}\PY{p}{,}\PY{p}{:}\PY{p}{,}\PY{p}{:}\PY{p}{]}\PY{p}{,} \PY{n}{XLAT}\PY{p}{[}\PY{l+m+mi}{0}\PY{p}{,}\PY{p}{:}\PY{p}{,}\PY{p}{:}\PY{p}{]}\PY{p}{)}
        
        \PY{n}{the\PYZus{}map}\PY{o}{.}\PY{n}{contourf}\PY{p}{(}\PY{n}{map2d\PYZus{}x}\PY{p}{,} \PY{n}{map2d\PYZus{}y}\PY{p}{,} \PY{n}{LLWS}\PY{p}{[}\PY{p}{:}\PY{p}{,}\PY{p}{:}\PY{p}{]}\PY{p}{)}
        
        \PY{n}{plt}\PY{o}{.}\PY{n}{colorbar}\PY{p}{(}\PY{p}{)}
        \PY{n}{plt}\PY{o}{.}\PY{n}{show}
\end{Verbatim}


    \begin{Verbatim}[commandchars=\\\{\}]
/usr/lib/python3/dist-packages/mpl\_toolkits/basemap/\_\_init\_\_.py:3644: VisibleDeprecationWarning: using a non-integer number instead of an integer will result in an error in the future
  xx = x[x.shape[0]/2,:]

    \end{Verbatim}

\begin{Verbatim}[commandchars=\\\{\}]
{\color{outcolor}Out[{\color{outcolor}3}]:} <function matplotlib.pyplot.show(*args, **kw)>
\end{Verbatim}
            
    \begin{center}
    \adjustimage{max size={0.9\linewidth}{0.9\paperheight}}{output_6_2.png}
    \end{center}
    { \hspace*{\fill} \\}
    
    \subsubsection{\texorpdfstring{ \emph{Compute "warning areas" where
\texttt{LLWS\ \textgreater{}\ 10} \&
Plot}}{ Compute "warning areas" where LLWS \textgreater{} 10 \& Plot}}\label{compute-warning-areas-where-llws-10-plot}

    \begin{Verbatim}[commandchars=\\\{\}]
{\color{incolor}In [{\color{incolor}4}]:} \PY{n}{THRESHOLD} \PY{o}{=} \PY{l+m+mf}{10.0}
        \PY{n}{masked\PYZus{}array} \PY{o}{=} \PY{p}{(}\PY{n}{LLWS} \PY{o}{\PYZgt{}} \PY{n}{THRESHOLD}\PY{p}{)}
        
        \PY{c+c1}{\PYZsh{} This is all we need to define a cylindrical map project}
        \PY{n}{the\PYZus{}map} \PY{o}{=} \PY{n}{bm}\PY{o}{.}\PY{n}{Basemap}\PY{p}{(}\PY{n}{projection}\PY{o}{=}\PY{l+s+s1}{\PYZsq{}}\PY{l+s+s1}{cyl}\PY{l+s+s1}{\PYZsq{}}\PY{p}{,}
                            \PY{n}{llcrnrlon}\PY{o}{=}\PY{n}{ll\PYZus{}lon}\PY{p}{,}
                            \PY{n}{llcrnrlat}\PY{o}{=}\PY{n}{ll\PYZus{}lat}\PY{p}{,}
                            \PY{n}{urcrnrlon}\PY{o}{=}\PY{n}{ur\PYZus{}lon}\PY{p}{,}
                            \PY{n}{urcrnrlat}\PY{o}{=}\PY{n}{ur\PYZus{}lat}\PY{p}{,}
                            \PY{n}{resolution}\PY{o}{=} \PY{l+s+s1}{\PYZsq{}}\PY{l+s+s1}{i}\PY{l+s+s1}{\PYZsq{}}\PY{p}{,}
                            \PY{n}{area\PYZus{}thresh}\PY{o}{=}\PY{l+m+mf}{100.0}\PY{p}{)}
        \PY{n}{the\PYZus{}map}\PY{o}{.}\PY{n}{drawcoastlines}\PY{p}{(}\PY{p}{)}
        
        \PY{k+kn}{import} \PY{n+nn}{matplotlib}\PY{n+nn}{.}\PY{n+nn}{colors}
        \PY{n}{cmap} \PY{o}{=} \PY{n}{matplotlib}\PY{o}{.}\PY{n}{colors}\PY{o}{.}\PY{n}{ListedColormap}\PY{p}{(}\PY{p}{[}\PY{l+s+s1}{\PYZsq{}}\PY{l+s+s1}{w}\PY{l+s+s1}{\PYZsq{}}\PY{p}{,}\PY{l+s+s1}{\PYZsq{}}\PY{l+s+s1}{r}\PY{l+s+s1}{\PYZsq{}}\PY{p}{]}\PY{p}{)}
        \PY{n}{plt}\PY{o}{.}\PY{n}{contourf}\PY{p}{(}\PY{n}{map2d\PYZus{}x}\PY{p}{,} \PY{n}{map2d\PYZus{}y}\PY{p}{,} \PY{n}{masked\PYZus{}array}\PY{p}{,} \PY{n}{cmap}\PY{o}{=}\PY{n}{cmap}\PY{p}{)}
        \PY{n}{plt}\PY{o}{.}\PY{n}{show}
\end{Verbatim}


\begin{Verbatim}[commandchars=\\\{\}]
{\color{outcolor}Out[{\color{outcolor}4}]:} <function matplotlib.pyplot.show(*args, **kw)>
\end{Verbatim}
            
    \begin{center}
    \adjustimage{max size={0.9\linewidth}{0.9\paperheight}}{output_8_1.png}
    \end{center}
    { \hspace*{\fill} \\}
    

    % Add a bibliography block to the postdoc
    
    
    
    \end{document}
